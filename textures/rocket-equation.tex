\documentclass[25pt, a2paper, landscape]{tikzposter}
\usepackage[utf8]{inputenc}

\definetitlestyle{sampletitle}{
}{
	\begin{scope}
		\draw (0,0) rectangle (0,0);
	\end{scope}
}
\usetitlestyle{sampletitle}

\begin{document}

	\maketitle

	\block{The Tsiolkovsky Rocket Equation}{
		\Huge $$ {\Delta}v = g_0 \cdot I_{sp} \cdot \ln \Big( \frac{m_{wet}}{m_{dry}} \Big) $$
	}
	\begin{columns}

		\column{0.5}
		\block{}{
			The Rocket Equation is what drives the design of every rocket. It says that
			how far out you can get on a rocket depends on the efficiency of the engines
			and the amount of fuel you bring. This means that if you want to get to (and
			stay in) space, you need a LOT of fuel, and VERY efficient engines.
		}

		\column{0.5}
		\block{Delta-v ($\Delta v$)}{
			How much a rocket can accelerate, and how far out it can go.
			A rocket must have at least 9.4 $\frac{km}{s}$ of $\Delta v$ to get to LEO,
			but only another 5.9 $\frac{km}{s}$ to land on the moon, and 7 $\frac{km}{s}$
			to get to Mars. Once you're in low Earth orbit, you're halfway to anywhere!
		}

	\end{columns}

	\begin{columns}

		%\column{0.25}
		%\block{$g_0$}{
		%	The gravitational constant, 9.82 $\frac{m}{s^2}$.
		%}

		\column{0.25}
		\block{Specific Impulse ($I_{sp}$)}{
			The rocket's fuel efficiency. Think of this like your car's MPG rating:
			the higher the rating, the farther you can get on a unit of fuel.
		}

		\column{0.25}
		\block{Fuel Mass Fraction ($\frac{m_{wet}}{m_{dry}}$)}{
			How much of the rocket's weight is fuel. Most rockets are around 95\% fuel.
		}

	\end{columns}
\end{document}
